\documentclass[a4paper,10pt]{article}
%\documentclass[a4paper,10pt]{scrartcl}
\usepackage{hyperref}
% \usepackage{caption}
\hypersetup{pdfborder={0 0 0}}
\usepackage[all]{hypcap}
\usepackage{default}
\usepackage{graphicx}
\usepackage{array}
\usepackage{amsmath}
\usepackage{listings}
\usepackage {apacite}
\lstset{language=Python,% general command to set parameter(s)
basicstyle =\small\ttfamily,          % print whole listing small
showstringspaces=false,
commentstyle=\ttfamily}

\title{Replication of Tyler et al. (2000)}
\author{Dr. Olivia Guest}
\date{\today}

\begin{document}
\maketitle
\section{Replicating the Model}
\subsection{Before Programming}
The focus of replication is the modelling account presented in the following journal article:
\begin{quote}
Tyler, L. K., Moss, H. E., Durrant-Peatfield, M. R., \& Levy, J. P. (2000). \textbf{Conceptual structure and the structure of concepts: A distributed account of category-specific deficits}. \textit{Brain and Language}, 75(2), 195-231.
\end{quote}
In terms of the code you have written so far, it is relatively straightforward to replicate what Tyler et al. (2000) have done provided you can find the values for the implemetation parameters and other modelling properties of the network. \emph{Before} you attempt any changes to your code answer the following questions to help clarify what needs changing: 
\begin{enumerate}

\item Describe the input and target patterns. (Do not bother typing these in yourself, I will give them to you.)

\ \\

\item What learning algorithm is being used?

\ \\

 \item List the values and names of the parameters to do with learning (e.g., the learning rate).

\ \\

\item Decsribe the network architecture: how many, how wide, and what types of layers are used?

\ \\

\item How are the weights initialised?

\ \\

\item Consider the connectivity of the model: are there any recurrent connections?

\ \\

\item What type and size of epochs are used?

\ \\

 \item Are there any ambiguities in the description of the model?
 
\ \\

\end{enumerate}

\subsection{Programming the Model}


Now apply the above to your code. If you need a fresh copy of the network you can always download it again. If you need the patterns I have supplied them as a \texttt{.csv} file and you may open them using the following function\footnote{The only reason that I changed line when calling \texttt{genfromtxt} and did not list each of the arguments within the brackets on the same line is to be tidy in this document. In your Python code, you do not need to have any space between the arguments --- they can all be on the same line, e.g., \texttt{x = f(a, b, c)}. You might also note how \texttt{genfromtxt} is a function we are calling from NumPy, hence why it is called \texttt{np.genfromtxt}, as previously in our code (near the very top) we imported the Numpy library using \texttt{import numpy as np}.}:

\begin{lstlisting}[language=Python]
Patterns = np.genfromtxt('tyler_patterns.csv',
                         delimiter=',',
                         dtype=int,
                         skip_header=1)
\end{lstlisting}

When you have applied all of the above, answer the following:

\begin{enumerate}

\item Would the network still learn with minor or dramatic changes to any of the above?

\ \\

\item Whatever your answer above what would that imply for the model? What about for the theory more broadly?

\ \\

\item How do you know if your replication has worked for the ``healthy'' model?

\ \\

\end{enumerate}

\section{Thinking about the Conceptual Structure Account}
In order to successfuly replicate the model you must understand both the theory and the implementation details of the model. In other words, it is important to understand what the Tyler et al. (2000) account is --- what explanations, hypotheses, predictions, does their framework provide?

Apart from training on the given patterns and evaluating its performance, which you are expected to do as part of this week's practical, it is important to be able to articulate what it is the model is and is not capturing.
So while you (re)read  the paper make sure to think about more cognitive, methodological, and theoretical, questions:
\begin{enumerate}


 \item What theory is this model part of?
 
\ \\

 
 \item What assumptions does the theory make? 

\ \\

 
 \item What assumptions does the model make? 

 
\ \\

 \item Which implementation details are central to the model and which are not? 

 
\ \\

 \item Which imprelemtation details are central to the theory and which are not? 

 
\ \\

 \item Does the model uniquely support a theory? Do results from the model lend support only to one theory? 

 
\ \\

 \item  What mechanism(s) is the model proposing?

 
\ \\

\item   What are the models predictions? 

\ \\

\item Can the model account for data it has not seen? If so, how --- if not, why not? 

\ \\

\end{enumerate}
\section{The Next Step}
In addition, to thinking about the original authors' work, consider what you can do with your knowledge of their model: 
\begin{enumerate}


 \item What kind of test(s) would you like to run on the model? If you want to recreate the exact results presented in Tyler et al. (2000), which involves removing weights, think about how this might be done. Otherwise, think of other useful ways of evaluating the model of healthy semantic cognition, might there be other variables you can play with other than lesioning (i.e., setting to zero) subsets of weights? This is up to your own judgement and you may perform any number of  experiments on the model motivated by your own curiosity and a sensible rationale. 
 
\ \\

\item In addtion, you may propose some extra features to be added to the model. What kind of augmentation to the model do you think might be useful or informative to attempt? This can be anything you think is appropriate; again a sensible motivation and a clear modelling goal are all that is required.
% \cite{attohokine99, hertz91}.
\end{enumerate}


\end{document}
