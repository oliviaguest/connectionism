\documentclass[a4paper,10pt]{article}
%\documentclass[a4paper,10pt]{scrartcl}
\usepackage{hyperref}
% \usepackage{caption}
\hypersetup{pdfborder={0 0 0}}
\usepackage[all]{hypcap}
\usepackage{default}
\usepackage{graphicx}
\usepackage{array}
\usepackage{amsmath}
\usepackage{listings}
\usepackage {apacite}
\lstset{language=Python,% general command to set parameter(s)
basicstyle =\small\ttfamily,          % print whole listing small
showstringspaces=false,
commentstyle=\ttfamily}

\title{Replication of Tyler et al. (2000)}
\author{Dr. Olivia Guest}
\date{\today}

\begin{document}
\maketitle
\section{Thinking about the Account}
In order to replicate the model you must understand both the theory and the implementation details of the model. In other words, it is important to understand what the Tyler et al. (2000) account is --- what explanations, hypotheses, predictions, does their framework provide?

Apart from training on the given patterns and evaluating its performance, which you are expected to do as part of this week's practical, it is important to be able to articulate what it is the model is and is not capturing.
So while you read through the paper make sure to think about more cognitive, methodological, and theoretical, questions --- some of these can only be answered when you have implemented your version of the model:
\begin{enumerate}


 \item What theory is this model part of?
 \item What assumptions does the theory make? 
 \item What assumptions does the model make? 
 \item Which implementation details are central to the model and which are not? 
 \item Which imprelemtation details are central to the theory and which are not? 
 \item Does the model uniquely support a theory? Do results from the model lend support only to one theory? 
 \item  What mechanism(s) is the model proposing?
 \item   What are the models predictions? 

\item Can the model account for data it has not seen? If so, how --- if not, why not? 

\end{enumerate}
\section{The Next Step}
In addition, to thinking about the original authors' work, consider what you can do with your knowledge of their model: 
\begin{enumerate}


 \item What kind of text would you like to run on the model? If you want to recreate the exact results presented in Tyler et al. (2000), which involves removing weights, think about how this might be done. Otherwise, think of other useful ways of evaluating the model of healthy semantic cognition, might there be other variables you can play with other than lesioning (i.e., setting to zero) subsets of weights? This is up to your own judgement and you may perfeorm any number of  experiments on the model motivated by your own curiosity and a sensible rationale. 
 
\item In addtion, you may propose some extra features to be added to the model. What kind of augmentation to the model would you propose? This is optional and can be anything you think is appropriate; again a sensible motivation and a clear modelling goal are all that is required.
% \cite{attohokine99, hertz91}.
\end{enumerate}


\end{document}
