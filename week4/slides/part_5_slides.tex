\documentclass{beamer}
% \usefonttheme{professionalfonts}
\usefonttheme{professionalfonts}

\usepackage{amsmath}
\usepackage{cancel}
%%%%%%%%%%%%%%%%%%%%%%%%% For making hand-outs %%%%%%%%%%%%%%%%%%%%%%%%%%%
% \documentclass[handout]{beamer}
% \usepackage{pgfpages}
% \pgfpagesuselayout{2 on 1}[a4paper,border shrink=5mm]
%%%%%%%%%%%%%%%%%%%%%%%%%%%%%%%%%%%%%%%%%%%%%%%%%%%%%%%%%%%%%%%%%%%%%%%%%%
\definecolor{inputred}{HTML}{c30e0e}
\definecolor{hiddenblue}{HTML}{2626c9}
\definecolor{outputgreen}{HTML}{008000}
\newcommand{\figheight}{0.72\textheight}
% \usepackage{eulervm}
\usepackage{default}
\usepackage{caption}
\usepackage{booktabs,mathptmx,siunitx}
% \usefonttheme{serif}
\graphicspath{{img/}}
\captionsetup{font=scriptsize, labelfont=scriptsize}


\usepackage{showexpl}



\begin{document}



\begin{frame}[fragile]

\centering\Huge Block Practical: Connectionist models and cognitive processes
\vfill \huge
\centering Part 5: \textbf{Writing the Report} \large
\vfill
\textit{
Olivia Guest }

\end{frame}


\begin{frame}[fragile]
\frametitle{Overview}
%\framesubtitle{Not boring, repetitive?}
        \  \\

\begin{itemize}[<+->]
<<<<<<< HEAD
\item Due by end of 0th week of Hilary term
=======
\item Due by end of 0th week of Hilary term, length: 2-3k words
>>>>>>> 26365ae74806393e7789897e9071165520742aad

\ \\

\ \\

 \item Aim to demonstrate that you understand the basics of the model and the theory
 
\ \\

 \ \\
 
\item Use your answers to the questions from last week to guide you


\end{itemize}


\end{frame}


\begin{frame}[fragile]
\frametitle{Sections of Report}

        \  \\

\begin{enumerate}[<+->]
\item Introduction

\ \\

\ \\

\item Methods

\ \\

\ \\

\item Results

\ \\

\ \\

\item Discussion


\end{enumerate}


\end{frame}


\begin{frame}
\frametitle{Introduction}
\framesubtitle{Which cognitive theory and what is the computational model?}
\begin{itemize}[<+->]


\item Explain: theory, what cognitive process/system you plan to model, the high-level architecture of model

\ \\

\ \\

\item Why does modelling help us?

\ \\

\ \\


\item What is the goal of the current model?

\ \\

\ \\


\item What part of cognition are we modelling and why did we chose this way?


\end{itemize}


\end{frame}

\begin{frame}
\frametitle{Methods}
\framesubtitle{The reader must be able to replicate the model!}
\begin{itemize}[<+->]


\item How did you replicate the model? Language, libraries, etc.

\ \\ 


 \item Table with parameters. Why did you chose these values?

\ \\


\item What kind of training algorithm? What training patterns?


\ \\


\item Did you try lots of settings? Document all of them. 

\ \\

\item Are certain values of the parameters essential? Can we absract over ranges of values?


\ \\


\item When did you chose to stop training? What was the range of the weights when you did? What was the error when you did?

\end{itemize}
\end{frame}

\begin{frame}
\frametitle{Results}
\framesubtitle{How do we test our model and how does it fare?}
\begin{itemize}[<+->]
\item What are the experiments you ran? Do you need graphs to demonstrate this?


\ \\

\ \\

\item How are you chosing to evaluate your model's output?


\ \\

\ \\

\item Are you looking at the hidden layer or the output layer, the sum squared error, something else?

\ \\ 

\ \\

\item Show all the evidence needed to help the reader understand your results.


\end{itemize}
\end{frame}


\begin{frame}
\frametitle{Discussion}
\framesubtitle{What have we learned?}
\begin{itemize}[<+->]

\item How does your work for into the broader picture of model and theory? 

\ \\

\ \\

\item Justify what's wrong if there is something wrong with implementation or with model or even with theory

\ \\

\ \\

 \item Is the original paper a good model?

\ \\

\ \\

\item Is your replication a good model and a good implementation? 


\end{itemize}

\end{frame}
\begin{frame}
\frametitle{Discussion}
\framesubtitle{What have we learned?}
\begin{itemize}[<+->]


\item Do you understand the original work more?

\ \\

\ \\


\item  Discuss the original work and how modelling it yourself (i.e., replicating) helped.

\ \\

\ \\

\item Why is replication useful?

\ \\

\ \\
\item Are there any experiments you would propose to carry out now you know more?
\end{itemize}
\end{frame}
\end{document}
